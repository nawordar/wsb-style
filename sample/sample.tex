\documentclass[]{wsbthesis}


\usepackage[english, polish]{babel}
\usepackage{csquotes}

% Paczka poprawiająca łamanie tekstu
\usepackage{microtype}

% Ustawienia dotyczące bibliografii
\usepackage[
	style=wsb,      % Ustaw styl bibliografii
	block=space,    % Zwiększ odstępy w bibliografii
	sortcites=true, % Sortuj pozycje zamiast układać je w kolejności występowania
	sorting=nty,    % Name-title-year — Imiona-tytuł-rok
	maxnames=5,     % Nie wypisuj wszystkich autorów, jeśli jest ich więcej niż 5
]{biblatex}
\addbibresource{sample.bib}

% Paczka do generowania przykładowego tekstu
\usepackage{bredzenie}

\usepackage[prevent-all]{widows-and-orphans}

% Dodaj wsparcie hiperlinków dla generowanego pdfa
\usepackage{bookmark}
\hypersetup{hidelinks}

% Dane do strony tytułowej
\author{Jan Kowalski}
\title{Projekt pracy inżynierskiej}
\university{Wyższa Szkoła Bankowa w Gdańsku}
\faculty{Wydział Ekonomii i Zarządzania w Gdyni}
\degree{Informatyka}
\albumnumber{12345}
% Praca napisana pod kierunkiem…
\supervisorgenitive{dr inż.\ Radowida Maciejewskiego}
\wyear{2020}


\begin{document}

% Jeśli adresy url nie mają być drukowane w foncie monospace
% \urlstyle{rm}

\maketitle{}
\printtableofcontent{}

% Numeruj strony od tego miejsca
\setcounter{page}{1}

%%%%%%%%%%%%%%
% Treść pracy
%%%%%%%%%%%%%%
\chapter{Przykładowy rozdział}
\bredzenie{1-5}
\section{Przykładowy podrozdział}
\bredzenie{6-10}

\chapter{Font o stałej szerokości}
Tekst o stałej szerokości nie jest prawidłowo justowany. Taki tekst należy wyrównać do lewej strony.
\begin{flushleft}
    \ttfamily\bredzenie{101-103}
\end{flushleft}

\chapter{Cytowanie}

\begin{enumerate}
    \item Książka
          \begin{enumerate}
              \item Książka nr 1 \autocite{kosieradzka_2012}
              \item Książka nr 2 \autocite{szoltysek_2016}
              \item Książka nr 3 \autocite{mateos_rosenberg_2011}
              \item Książka nr 4 \autocite{banaszak_2016}
          \end{enumerate}

    \item Rozdział w~książce \autocite{leszek_2016}

    \item Artykuł w czasopiśmie
          \begin{enumerate}
              \item Artykuł nr 1 \autocite{zawila_2016}
              \item Artykuł nr 2 \autocite{kisman_2014}
          \end{enumerate}

    \item Artykuł z czasopisma dostępnego przez sieć Internet \autocite{kompa_2016}

    \item Artykuły i raporty publikowane w sieci Internet, w których wskazany jest autor
          \begin{enumerate}
              \item Artykuł nr 1 \autocite{panetta_2017}
              \item Artykuł nr 2 \autocite{beauchamp_2017}
          \end{enumerate}

    \item Artykuły i raporty publikowane w sieci Internet, w których nie ma autora (autorem jest instytucja publikująca)
          \begin{enumerate}
              \item Artykuł nr 1 \autocite{gus_2016}
              \item Artykuł nr 2 \autocite{gus_2016a}
          \end{enumerate}

    \item Źródła internetowe (blogi, video, strony WWW)
          \begin{enumerate}
              \item Źródło nr 1 \autocite{gpw_2018}
              \item Źródło nr 2 \autocite{mit_2018}
              \item Źródło nr 3 \autocite{oreilly_2017}
              \item Źródło nr 4 \autocite{easypark_2017}
          \end{enumerate}

    \item Akty prawne
          \begin{enumerate}
            \item Akt nr 1 \autocite{dzu_2009_1_3}
            \item Akt nr 2 \autocite{2010_13_ue_1}
          \end{enumerate}
\end{enumerate}

\printbibliography[heading=bibintoc]

\end{document}
